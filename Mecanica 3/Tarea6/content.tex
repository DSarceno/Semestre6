

\section{Problema 1}
Dada la fuerza gravitacional
\begin{displaymath}
	\vec{F} = -G\frac{Mm}{r^3} \vec{r} \nonumber
\end{displaymath}
y la condición $M \gg m$, se tiene:
\begin{enumerate}[a)]
	\item Dado que la fuerza (en coordenadas polares) se define como $-\pdv{U}{r} = F$ (Magnitud), se tiene que la energía potencial, al integrar, es.
		$$\boxed{U = -\frac{GMm}{r}}$$
	\item Dada la definición general de energía cinética para coordenadas ortogonales
		$$T = \frac{1}{2} m\qty(h_1 ^2 \dot{q}_1 ^2 + h_2 ^2 \dot{q}_2 ^2 + h_3 ^2 \dot{q}_3 ^2 + \cdots)$$
		Para coordenadas polares, se tiene los factores de escala $h_r = 1; \, h_\theta = r$, sustituyendo y sabiendo que $\dot{r} = 0$, entonces
		$$\boxed{T = \frac{1}{2} m r\dot{\theta} ^2}$$
	\item Dada la energía cinética y potencial del planeta, se tiene que el lagrangiano es:
		$$\boxed{\lagran = \frac{1}{2} m r\dot{\theta} ^2 + \frac{GMm}{r}}$$
	\item Para el hamiltoniano, se tienen los momentum conjugados:
		$$
			\left\{\begin{array}{c}
				p_r = 0 \\
				p_\theta = mr\dot{\theta}
			\end{array}\right.
		$$
		Con esto, sustituyendo en la definición de hamiltoneano, se tiene:
		$$\boxed{H = \frac{1}{2} m r\dot{\theta} ^2 - \frac{GMm}{r}}$$
	\item Dado que se tienen únicamente dos grados de libertad, se tienen dos ecuaciones de Euler-Lagrange, de modo que
		$$\pdv{\lagran}{r} + \dv{t} \pdv{\lagran}{\dot{r}} = 0 \, \rightarrow \, \boxed{r^2 \dot{\theta} ^2 = 2GM}$$,
		para la otra coordenada
		$$\pdv{\lagran}{\theta} + \dv{t} \pdv{\lagran}{\dot{\theta}} = 0 \, \rightarrow \, \boxed{\ddot{\theta} = 0}$$
\end{enumerate}


\section{Problema 2}
Dado el lagrangiano y que se conseva dadas las transformaciones, entonces se tiene:
\begin{displaymath}
	\lagran \qty(t,q_1,\ldots ,q_n ,\dot{q}_1, \ldots ,\dot{q}_n) = \lagran \qty(t + \delta t,q_1 + \delta q_1,\ldots ,q_n + \delta q_n ,\dot{q}_1, \ldots ,\dot{q}_n)
\end{displaymath}
Esto implica que $\pdv{\lagran}{t} = 0$ y $\pdv{\lagran}{q_i} = 0$, dada la primera igualdad se tiene
	$$\dv{H}{t} = -\pdv{\lagran}{t} = 0,$$
el hamiltoniano se conserva; además, dada la segunda igualdad
	$$\dv{t} \pdv{\lagran}{\dot{q}_i} = \dv{p_i}{t} = 0,$$
por lo que, los momentum conjugados se conservan. Dadas ambas conclusiones, y que $\delta t$ y $\delta q_i$ no dependen del tiempo, se tiene claro, aplicando el Teorema de Noether, que
	$$\dv{J}{t} = \dv{t} \qty[\sum_i ^n p_i \delta q_i - H\delta t] = 0;$$
lo que, en efecto, implica que $\displaystyle\sum_i ^n p_i \delta q_i - H\delta t = \text{cte}$. 


\section{Problema 3}
Dada la relación $q_1 ' = q_1 \cos{\theta} + q_2 \sin{\theta}$ y $q_2 ' = -q_1 \sin{\theta} + q_2 \cos{\theta}$. Por lo que, los deltas quedan tal que: $\delta q_1 = q_1 (\cos{\theta} - 1) + q_2 \sin{\theta}$ y $\delta q_2 = -q_1 \sin{\theta} + q_2 (\cos{\theta} - 1)$. Multiplicando y dividiendo entre $\theta$, luego tomando el límite dentro del parentesis cuando $\theta \to 0$, lo que da:
	$$\delta q_1 = \qty[\lim_{\theta \to 0} \qty(\frac{q_1 (\cos{\theta} - 1)}{\theta} + \frac{q_2 \sin{\theta}}{\theta})] \theta ,$$
	$$\delta q_2 = \qty[\lim_{\theta \to 0} \qty(\frac{q_2 (\cos{\theta} - 1)}{\theta} - \frac{q_1 \sin{\theta}}{\theta})] \theta.$$

Con esto se concluye que $\delta q_1 = q_2 \theta$ y $\delta q_2 = q_1 \theta$. Ahora, por el teorema de Noether ($\delta t = 0$):
	$$\dv{t} \qty[p_1 q_2 \theta - p_2 q_1 \theta] = 0;$$
por lo tanto, $J = p_1 q_2 - p_2 q_1 = \text{cte}.$


\section{Problema 4}






























